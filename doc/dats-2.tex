
\documentclass[11pt]{article}
\usepackage[top=1in, left=1.5in, right=1.5in]{geometry}
\usepackage[parfill]{parskip}
\usepackage{fancyvrb}
\usepackage{amsmath}
\usepackage{mathtools}
\usepackage{hyperref}
\usepackage{multicol}
\usepackage{xcolor}
\usepackage{fancyvrb}

\title{\textbf{The Dats\thanks{\url{github.com/harieamjari/dats/}} language v2.0.0}}
\newcounter{numpar}
\newcommand{\np}[1][]{%
  \par
  \refstepcounter{numpar}%
  \noindent
  \makebox[0pt][r]{%
    \makebox[0pt][l]{{\footnotesize\thenumpar}}%
    \qquad
  }%
  #1%
  \ignorespaces
}
\newcommand\textgray[1]{\fcolorbox{gray!80}{gray!10}{\texttt{#1}}}

\let\stdsection\section
\renewcommand\section{\clearpage\setcounter{numpar}{0}\stdsection}
\let\stdsubsection\subsection
\renewcommand\subsection{\setcounter{numpar}{0}\stdsubsection}
\let\stdsubsubsection\subsubsection
\renewcommand\subsubsection{\setcounter{numpar}{0}\stdsubsubsection}
\hypersetup{pdftitle={The Dats Language},
            pdfcreator={Al-buharie Amjari},
            bookmarks=true,
            bookmarksnumbered=true,
            pdfpagelabels=true,
            pdfpagemode=UseOutlines,
            pdfstartview=FitH,
            linktocpage=true,
            colorlinks=true,
            linkcolor=blue,
            plainpages=false}

\begin{document}
\setcounter{tocdepth}{4}
\setcounter{secnumdepth}{4}
\maketitle

\begin{Verbatim}[frame=single, label=The "Hello World" of Dats]
      staff foo {
        n 4, c4;
        n 4, d4;
        n 4, e4;
      }

      main {
        pcm16 bar = synth.psg(foo);
        write("t.wav", bar);
      }
\end{Verbatim}
\vspace{1in}

\begin{abstract}
This is part of the Dats v2.0.0 documentation\footnote{\url{github.com/harieamjari/dats/blob/main/doc/dats-2.tex}} and is a draft. This document may be outdated in the future so readers are adviced instead to follow 1, although this draft should be able to provide the syntax and able the user to extrapolate from these to
start writing a music composition in Dats language.

Further revision of this draft may be publish by the author from time to time.
\end{abstract}

\tableofcontents

\section{Introduction}
\np The Dats language is a text-based programming language use for music composition.  
It intents to promote the readability of ASCII music sheets and the compilation of these into
 a soundfile.
This language is fairly new and still in experimental so other features such trills, tie,
repeat, glissando, are not implemented but are considered to be in the future. It preserves the
concepts used in engraving music sheets via a syntatically similar one. This is
recognized by the use of similar terminologies like:

\begin{multicols}{2}
\begin{itemize}
\renewcommand{\labelitemi}{--}
\item staff
\item repeat
\item note
\item rest
\item octave
\item bpm (beats per minute)
\item semitone
\end{itemize}
\end{multicols}

\np Polyphony in

\section{Terms, definitions and symbols}

\begin{enumerate}
\item Musical note

Literal "Musical note" which plays a note for some duration.

\item Musical rest

Literal "Musical rest" which plays silence for some duration.

\item Note

A literal express by the RegEx: \verb+[a-g](#|b)?[0-9]+, and is used
in the parameter of a musical note as \textgray{<note>}.

\end{enumerate}

\section{The basics}

\np Toward playing music, there must be series of musical notes written for the player to read.
For dats, a musical note is declared with the keyword \verb+n+. \verb+n+ must contain two arguments,
a length and a note, each are separated by a comma:


\begin{Verbatim}[frame=single]
       n <length>, <note>;
\end{Verbatim}


\np And example of such declaration is:
\begin{Verbatim}[frame=single]
       n 4, c4;
\end{Verbatim}

This is a declaration of musical note of length 1/4, playing the note c4. The length
is just the denominator of how much measure the musical note took.

\np Such musical notes, (and musical rests), are always declared inside a \textit{staff block}:

\begin{Verbatim}[frame=single]
      staff foo {
        n 4, c4;
      }
\end{Verbatim}

\np Now we have a staff, what else do we need? A `main`. A main is basically
where you do your operations on staffs and synthesize, producing sounds.

\np First, we process the staff using a synthesizer. The output of this synthesizer
is then stored into a variable, usually a type of `pcm16`:

\begin{Verbatim}[frame=single]
      staff foo {
        n 4, c4;
      }
    
      main {
        pcm16 bar = synth.kpa(foo);
        write("w.wav", bar);
      }
\end{Verbatim}

\np This is then written into a file named, "w.wav". (Currently, Dats only supports writing wav files.)

\np To append one or more pcm16 to another pcm16, you may simply just add a comma, followed by the pcm16
you were appending with:

\begin{Verbatim}[frame=single]
      pcm16 tr1 = synth.kpa(/* some staff */);
      pcm16 tr2 = synth.kpa(/* some staff */);
      pcm16 tr3 = tr1, tr2, tr2;
\end{Verbatim}

\setcounter{numpar}{0}
\section{Environment}
\np A Dats interpreter executes each statements from a Dats file sequentially. These
statements are always terminated by a semicolon.

\np The start of a Dats file is called, \verb+main+.
\section{Language and notation}
\subsection{Lexical elements}
\subsubsection{Character sets}

The Dats v2.0.0 language supports the 26 \textit{uppercase letters} and 26
\textit{lower case letter} of the \textit{Latin alphabet}:

\begin{verbatim}
       A B C D E F G H I J K L M
       N O P Q R S T U V W X Y Z
       a b c d e f g h i j k l m
       n o p q r s t u v w x y z
\end{verbatim}

the 10 {decimal digits}:
\begin{verbatim}
       0 1 2 3 4 5 6 7 8 9
\end{verbatim}

and the miscs symbols:
\begin{verbatim}
       - + * / ( ) { } [ ] / " ; = .
\end{verbatim}

\subsubsection{Keywords}

\np Keywords: one of

\begin{verbatim}
       attack        bpm          pcm16
       decay         filter
       main          mix
       octave        read
       release       semitone
       sustain       synth
       volume        write
\end{verbatim}

\subsubsection{Identifier}


\subsubsection{Note}

\np \textbf{Syntax}

\begin{verbatim}
<note> : [a-g](#|b)?[0-9] <symbols>
       | <note> "," <note>
<symbols> : <staccatto>
          | <staccatissmo>
          |
          ;
<staccatto> : "."
            ;
<staccatissimo> : "_"
                ;
\end{verbatim}

\subsubsection{Length}

\subsection{\texttt{staff}}
\np The \textgray{staff} keyword signifies the beginning of a staff. It is then followed by an identifier and \verb+{+. A staff consist of
notes and rests which is declared by using \verb+n+, or \verb+r+, inside the \textit{staff block}.
The staff, is terminated by \verb+}+. The, \verb+n+, or, \verb+r+, which is declared inside the \textit{staff block} both takes an argument.

\subsubsection{\texttt{n}}

\np The \verb+n+ keyword signifies the beginning of a musical note. It must be declared
inside the staff block.
It takes 2 parameters; the \textgray{<length>} and the \textgray{<note>} of the
musical note, each separated by a comma. It must end with a semicolon: 

\begin{Verbatim}[frame=single]
      n <length>, <note>;
\end{Verbatim}

\paragraph{\texttt{<note>}} The parameter \textgray{<note>} consist of a \verb+key+ and an \verb+octave+. It shall be in the form of \verb+[a-g](#|b)?[0-9]+:

\begin{Verbatim}[frame=single]
  c4 // [abcdefg](b|#)?[0-9]
  ||
  |`--->octave 4
  v
 key 'c'
\end{Verbatim}
The above has a key 'c' at octave 4.

\np To play a whole note ringing at c4:
\begin{Verbatim}[frame=single]
      staff {
        n 1, c4;
      }
\end{Verbatim}

a half of a note ringing at d4:

\begin{Verbatim}[frame=single]
      staff {
        n 2, d4;
      }
\end{Verbatim}

and a quarter of a note ringing at e4:
\begin{Verbatim}[frame=single]
      staff {
        n 4, e4;
      }
\end{Verbatim}

\np One can derive the length of the note from the denominator of these notes:
\verb+1/1 1/2 1/4+

\subparagraph{dyad}
The parameter \textgray{<note>} may be entered for a number of times:

\begin{Verbatim}[frame=single]
      staff {
        n 1, c4 e4 g4;
      }
\end{Verbatim}

\np The example above plays the C major chord at root position as whole note. (Integer overflow is not
protected; you might hear undesirable sounds.)

\paragraph{\texttt{<length>}} For argument, \textgray{<length>}, it may be in the form of integer or floating point number:
\label{nlength}

\begin{Verbatim}[frame=single]
      n 4.0, <note>;
      n 4, <note>;
\end{Verbatim}

\np To simulate a tie, use \verb-+-

\begin{Verbatim}[frame=single]
      n 4+4, 440; /* is equivalent to: */
      n 2, 440; 
\end{Verbatim}

\np How the former is equivalent to \textgray{n 2, 440;} might
not make sense as, $4+4$ is mathematically equal to 8. Instead
of treating this as an addition operation which sums up two
operands (which is confusing), consider to pronouce it as
\textit{four-tie-four} instead of \textit{four-plus-four}.
 
\textbf{NOTE} Other arithmetic operations are prohibited.

\np The paramater, \textgray{<length>}, will take any real positive integer (as long as it's in the range of int),
so a length of \verb+3+ will work (equivalent to 1/3 of a note), \verb+5+ will work (equivalent to 1/5
of a note) and others.

\np To play a 1/8 note triplet, one may need to do some math:

\begin{align}
	\shortintertext{Multiply 1/8 with 2:}
	\frac{1}{8}2 = \frac{2}{8} \Rightarrow \frac{1}{4} \\
	\shortintertext{divide the answer with 3 (3 for triplet):}
	\frac{1}{4(3)} = \frac{1}{12}
\end{align}

and use the denominator, \verb+12+, as the value of the length:
\begin{Verbatim}[frame=single]
      staff {
        n 12, c4;
        n 12, d4;
        n 12, e4;
      }
\end{Verbatim}

This process can be repeates for 1/8 note quadruplet, but instead of dividing
with 3 in step (2), it's divided with 4 (4 for quadruplet) and so on.

\np Single and double dotted notes are implemented but three dots were not. To use dots, you may need to use a period. An example below is a 1/4 dotted note:
\begin{Verbatim}[frame=single]
      staff {
        n 4., c4;
        n 4..,d4;
      }
\end{Verbatim}

\np The key from A-G is supported, and the octave it takes may be any positive integer, but
it should not exceed the range of \verb+int+.

\np To use accidentals, use \verb+#+ to raise the note a semitone and use \verb+b+ to lower the
note a semitone:
\begin{Verbatim}[frame=single]
      staff {
        n 1, c#4;
        n 2, db4;
      }
\end{Verbatim}

\np Unlike the parameter \textgray{length>}, arithmetic operation is illegal
to perform on a \textgray{<note>}. so \textgray{<note>+<note>} is prohibited.

- To raise the note a semitone, append \verb+#+ to the first letter of the note, and to lower
the note a semitone, append \verb+b+ to the first letter of the note.

- To perform a staccato, append a \verb+.+ to every ending of a note.

\begin{Verbatim}[frame=single]
  n 4, c4.;
\end{Verbatim}

- To play a chord or a dyad, like `c4 e4 g4`, one just needs to write as it is:

\begin{Verbatim}[frame=single]
  n 4, c4 e4 g4;
\end{Verbatim}

\textbf{NOTE} If all these notes is played as staccato, append,
\textgray{.}, to them, so, it becomes, \textgray{c4. e4. g4.}

\subsubsection{\texttt{r}}

\np The \verb+r+ keyword signifies the beginning of a musical rest. It must be
declared inside the \textit{staff block}. It takes one parameter; the
length of the rest:

\begin{Verbatim}[frame=single]
      r <length>;
\end{Verbatim}

\np The syntax of \textgray{<length>} is similar to the definition in 
\autoref{nlength}.

\subsubsection{Extensions}

\subsection{\texttt{pcm16}}
\np This is is a signed 16-bit PCM. A declaration of variable of type,
\verb+pcm16+, must be immdediately followed by its identifier, followed by
an assignment to one of synthesizers taking a variable of type, \verb+staff+. 
\begin{Verbatim}[frame=single]
      pcm16 foo = synth.psg(pol);
\end{Verbatim}

or other variable of type ,\verb+pcm16+.
\begin{Verbatim}[frame=single]
     pcm16 foo = synth.psg(pol);
     pcm16 bar = foo;
\end{Verbatim}

\np To append these supported types, a comma must be added, followed by an rvalue:
\begin{Verbatim}[frame=single]
      pcm16 foo = synth.psg(bar), synth.psg(bar);
\end{Verbatim}


\subsubsection{Extensions}

\np The length of the rest is semantically similar to one of that length
of the note. Refer to the \verb+n+ keyword.

\subsection{Comments}
\np Just like in the C programming language, you can use a double slash (solidus if you like), \verb+//+.
It eats the letters/words in its front until it reaches a newline.
\begin{Verbatim}[frame=single]
      staff {
        n 1, c4; //EAT MEEE!!!
      }
\end{Verbatim}

\subsection{Formatting}
\np To preserve the readability of the text, it is a recommended practice to add a comment for every
beginning measure, regardless of the time signature. Below is a 4/4 time signature:
\begin{Verbatim}[frame=single]
      staff {
        // Measure 1
        n 1, c4;
  
        // Measure 2
        n 4, d4;
        n 4, e4;
        n 4, f4;
        n 4, g4;
  
        // Measure 3
        n 2, a4;
        n 2, b4;
      }
\end{Verbatim}

\section{Synths}

\section{Filters}

\section{EBNF}
\section*{Afterword}

Currently, even myself is having hard time transcribing music into Dats. It can be time consuming to debug the ASCII music sheet whenever a note was place 
incorrectly and a pain in the butt.



\end{document}
