
\documentclass{article}
\usepackage[top=0.4in]{geometry}
\usepackage{fancyvrb}
\usepackage{amsmath}
\usepackage{mathtools}
\usepackage{hyperref}

\title{\textbf{The Dats\thanks{\url{github.com/harieamjari/dats/}} language v2.0.0}}
\author{Al-buharie Amjari}

\begin{document}
\maketitle

\abstract
This is part of the Dats v2.0.0 documentation\footnote{\url{github.com/harieamjari/dats/blob/master/doc/dats-2.tex}} and is a draft. This document may be outdated in the future so readers are adviced instead to follow 1, although this draft should be able to provide the syntax and able the user to extrapolate from these to
start writing a music composition in Dats language.

Further revision of this draft may be publish by the author from time to time.


\section{Introduction}

\indent The Dats language is a text-based programming language use for music composition.  
It intents to promote the readability of ASCII music sheets and the compilation of these into
 a soundfile.
This language is fairly new and still in expiremental so other features such trills, tie,
repeat, glissando, are not implemented but are considered to be.



\section{Environment}
 \paragraph{} A Dats interpreter executes each statements from a Dats file sequentially. These
statements are always terminated by a semicolon.

\paragraph The interpreter 

\section{Language}
\subsection{Types}
\subsubsection{staff}
\paragraph{} The \verb+staff+ keyword signifies the beginning of a staff. It is then followed by an identifier and \verb+{+. A staff consist of
notes, which is declared by using \verb+n+, and or rest, which is declared by using \verb+r+, inside the \textit{staff block}. The staff, is terminated by \verb+}+. The, \verb+n+, or, \verb+r+, which is declared inside the \textit{staff block} both takes an argument.

\paragraph{} \verb+n+, takes two arguments; the length and the frequency the note will be playing:
\begin{Verbatim}[frame=single]
      n <length>, <frequency>;
\end{Verbatim}

\paragraph{} For argument, length, it may be in the form of integer or floating point number:
\begin{Verbatim}[frame=single]
      n 4.0, <frequency>;
      n 4, <frequency>;
\end{Verbatim}
\paragraph{} For argument, \verb+frequency+, it may be in the form of \verb+[a-g](#|b)?[0-9]+, a floating
point number, or an integer:

\begin{Verbatim}[frame=single]
      n 4, 440;
      n 4, 440.0;
      n 4, a4;
\end{Verbatim}

\paragraph{} Add operation may be performed in argument, length, to simulate a tie:

\begin{Verbatim}[frame=single]
      n 4+4, 440; /* is equivalent to: */
      n 2, 440; 
\end{Verbatim}
\textbf{NOTE} Other arithmetic operations are prohibited.

\subsubsection{pcm16}
\paragraph{} This is is a signed 16-bit PCM. A declaration of variable of type,
\verb+pcm16+, must be immdediately followed by its identifier, followed by
an assignment to one of synthesizers taking a variable of type, \verb+staff+. 
\begin{Verbatim}[frame=single]
      pcm16 foo = synth.psg(pol);
\end{Verbatim}

\paragraph{} Or other variable of type ,\verb+pcm16+.
\begin{Verbatim}[frame=single]
     pcm16 foo = synth.psg(pol);
     pcm16 bar = foo;
\end{Verbatim}

\paragraph{} Append operation 
\subsection{NOTE}

The \verb+NOTE+ keyword signifies the beginning of a note. It must be declared
inside the main block.
It takes 2 parameters; the length and the key of the note. It must end with
a semicolon: 
\begin{Verbatim}[frame=single]
      NOTE <length> <key>;
\end{Verbatim}

To play a whole note ringing at C4:
\begin{Verbatim}[frame=single]
      BEGIN
      NOTE 1 C4;
      END
\end{Verbatim}
\newpage
A half of a note ringing at D4:
\begin{Verbatim}[frame=single]
      BEGIN
      NOTE 2 D4;
      END
\end{Verbatim}

And a quarter of a note ringing at E4:
\begin{Verbatim}[frame=single]
      BEGIN
      NOTE 4 E4;
      END
\end{Verbatim}

One can derive the length of the note from the denominator of these notes:
\verb+1/1 1/2 1/4+

\subsubsection{Extensions: dyad}
The parameter \verb+<key>+ may be entered for a number of times:
\begin{Verbatim}[frame=single]
      BEGIN
      NOTE 1 C4 E4 G4;
      END
\end{Verbatim}

The example above plays the C major chord at root position as whole note. (Integer overflow is not
protected; you might hear undesirable sounds.)

\subsubsection{Extensions: others}
The paramater, \verb+length+, will take any real positive integer (as long as it's in the range of int),
so a length of \verb+3+ will work (equivalent to 1/3 of a note), \verb+5+ will work (equivalent to 1/5
of a note) and others. \newline

To play a 1/8 note triplet, one may need to do some math:

\begin{align}
	\shortintertext{Multiply 1/8 with 2:}
	\frac{1}{8}2 = \frac{2}{8} \Rightarrow \frac{1}{4} \\
	\shortintertext{divide the answer with 3:}
	\frac{1}{4(3)} = \frac{1}{12}
\end{align}

And use the denominator, \verb+12+, as the value of the length:
\begin{Verbatim}[frame=single]
      BEGIN
      NOTE 12 C4;
      NOTE 12 D4;
      NOTE 12 E4;
      END
\end{Verbatim}

Single and double dotted notes are implemented but three dots were not. To use dots, you may need to use a period. An example below is a 1/4 dotted note:
\begin{Verbatim}[frame=single]
      BEGIN
      NOTE 4.  C4;
      NOTE 4.. D4;
      ENDD
\end{Verbatim}

The key from A-G is supported, and the octave it takes may be any positive integer, but
it should not exceed the range of \verb+int+. \newline

To use accidentals, use \verb+#+ to raise the note a semitone and use \verb+b+ to lower the
note a semitone:
\begin{Verbatim}[frame=single]
      BEGIN
      NOTE 1 C#4;
      NOTE 2 Db4;
      END
\end{Verbatim}

\subsection{REST}
The \verb+REST+ keyword signifies the beginning of a rest. It must be declared inside
the main block.
The keyword takes 1 parameter; the length of the rest. It must end with a semicolon:
\begin{Verbatim}[frame=single]
      REST <length>;
\end{Verbatim}

\subsubsection{Extensions}

The length of the rest is semantically similar to one of that length
of the note. Refer to the \verb+NOTE+ keyword.

\subsection{BPM}

The \verb+BPM+ keyword sets the BPM of the music. It must be declared inside the main block.
It takes 1 parameter; the BPM of the music in integer or decimal form. It must end
with a semicolon:
\begin{Verbatim}[frame=single]
      BPM <BPM>;
\end{Verbatim}

\subsubsection{Extensions}

The \verb+BPM+ keyword may be use to change the BPM of the music midway:
\begin{Verbatim}[frame=single]
      BEGIN
      BPM 80;
      NOTE 8 C4;
      NOTE 8 D4;

      BPM 130;
      NOTE 8 E4;
      NOTE 8 F4;
      END
\end{Verbatim}

If \verb+BPM+ is not set, the default is set to be 120.

Regex: \verb@[0-9]+(\.[0-9]+)?@
\subsection{use}

The \verb+use+ keyword sets the soundfont of the music. It must be declared before the main block begins.
We may need to call this block the "prerequisite".
It takes two parameter; the location of the library and the name of the soundfont
or more technically, the name of the function:
\begin{Verbatim}[frame=single]
      use <library> <soundfont>
\end{Verbatim}

This opens the shared library with \verb+dlopen(3)+\footnote{\url{https://www.man7.org/linux/man-pages/man3/dlopen.3.html}}.

If this is not set, it uses the default library, \verb+libpsg.so+ and uses the function \verb+guitar+\footnote{It employs \href{wikipedia.org/wiki/Karplus\%E2\%80\%93Strong_string_synthesis}{karplus-strong algorithm} to synthesize the guitar-like sound.}

\subsubsection{Extensions}
For creating your own soundfont, please refer to the file, "Plugin.md" (Which is unfinished yet).

\section{Comments}
Just like in the C programming language, you can use a double slash (solidus if you like), \verb+//+.
It eats the letters/words in its front until it reaches a newline.
\begin{Verbatim}[frame=single]
      BEGIN
      NOTE 1 C4; //EAT MEEE!!!
      END
\end{Verbatim}

\section{Formatting}
To preserve the readability of the text, it is a recommended practice to add a comment for every
beginning measure, regardless of the time signature. Below is a 4/4 time signature:
\begin{Verbatim}[frame=single]
      BEGIN

      // Measure 1
      NOTE 1 C4;

      // Measure 2
      NOTE 4 D4;
      NOTE 4 E4;
      NOTE 4 F4;
      NOTE 4 G4;

      // Measure 3
      NOTE 2 A4;
      NOTE 2 B4;

      END
\end{Verbatim}

\section*{Afterword}

Currently, even myself is having hard time transcribing music into Dats. It can be time consuming to debug the music sheet whenever a note was place 
incorrectly and a pain in the butt.

\textbf{Composition created with Dats:}

1. \url{facebook.com/groups/427631411331889/permalink/837013050393721}

\end{document}
